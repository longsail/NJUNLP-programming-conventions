% !TEX TS-program = XeLaTeX
% !TEX encoding = UTF-8 Unicode

\chapter{命名规范}
\label{chap01}

从建立项目到书写每一行代码,无时无刻不需要进行各种命名。命名规范给出了各类命名任务的基本命名规则,其目的是使程序具有更好的可读性,从而更易于理解、重用和维护。下面首先给出不受限于语言的通用命名规范,再针对部分语言中的特殊情况(如Java中的包,C\#中的名空间等)给出相应的规范。

\section{通用命名规范}

\subsection{基本规则}
常用的命名单位包括类、接口、变量、常量、方法等。命名应采用大小写混合的方式,
类名和接口名首字母大写,变量名和方法名首字母小写;命名由多个单词构成时,后续单词首字母大写。
变量名不应该以下划线或\$开头。常量的命名采用全部字母大写,单词间用下划线隔开。

示例如下:
\begin{enumerate}

\item \textcolor{blue}{类名~~~~~~~~~~~~~~~~~~~~~~~~~Raster~~~~~~~~~~~~~~~~~~~~~~~~~ImageSprite}
\item \textcolor{blue}{接口名~~~~~~~~~~~~~~~~~~~~~Storing~~~~~~~~~~~~~~~~~~~~~~~~RasterDelegate}
\item \textcolor{blue}{方法名~~~~~~~~~~~~~~~~~~~~~getMax()~~~~~~~~~~~~~~~~~~~~~runFast()}
\item \textcolor{blue}{变量名~~~~~~~~~~~~~~~~~~~~~myWidth~~~~~~~~~~~~~~~~~~~~~fruitShape}
\item \textcolor{blue}{常量名~~~~~~~~~~~~~~~~~~~~~MIN\_WIDTH~~~~~~~~~~~~~MAX\_WIDTH}
\end{enumerate}
%使用模版的第一步当然是修改您的个人信息。与个人信息有关的内容位
%于~{/preface/cover.tex}~文件中。对照着模版内容改就好了,没有什么难度。填
%写专业、姓名和导师的时候注意添加适当空格,也就$\sim$字符,以保持段落对齐。
%这里的默认完成时间是最后一次编译main.tex的日期。

\subsection{命名的选择}
命名应在不使用注释的情况下也能较好的体现出作者的意图,应该告诉代码阅读人员,
它们为什么会存在,做什么事情。类、接口、变量、常量的命名应使用名词,方法的命名应使用动词。
名称必须是可读的,这样会方便讨论。

示例如下:
\begin{enumerate}
\item  \textcolor{blue}{消逝的时间~~~~~~~~~~~~~~~elapsedTime}
\item  \textcolor{blue}{生成时间戳~~~~~~~~~~~~~~~generateTimeStamp}
\end{enumerate}

\subsection{命名的长度}
避免使用单字命名。在我们阅读或者修改代码的过程中,有时候需要查找变量名或者常量名,然而单字母名称和数字常量很难在一个代码文件中找到,这也造成了相当程度的不便。      

若变量或常量可能在代码中多处用到,则应赋予其便于搜索的名称,建议命名长度与作用域大小相对应,即作用域越大命名越长。同时,避免命名超长(原则上不应该超过20个字符)。

示例如下:
\begin{enumerate}
%\begin{lstlisting}
\item  \textcolor{blue}{变量名~~~~~~~~~~~~~~~~~~~~~~~~~realDaysPerYear}
\item  \textcolor{blue}{常量名~~~~~~~~~~~~~~~~~~~~~~~~~WORK\_DAYS\_PER\_WEEK}
%\end{lstlisting}
\end{enumerate}

\subsection{慎用缩写}
请尽量少使用意义不明确的缩写,如果要用到缩写,尽量按照大家公认的缩写名称来使用,并在使用时用注释进行说明。

示例如下:
\begin{enumerate}
%\begin{lstlisting}
\item  \textcolor{blue}{缩写规则~~~~~~~~~~~~~~~~~~~~~~~NO.代表number}
\item  \textcolor{blue}{缩写规则~~~~~~~~~~~~~~~~~~~~~~~ID.代表identification}
%\end{lstlisting}
\end{enumerate}

\section{语言特有的命名}

\subsection{Java}

\begin{enumerate}
\item 包

包的命名应全部使用小写的ASCII字母。由于互联网上的域名是不会重复的,所以一般采用在互联网上的域名作为包的惟一前缀。
NLP组的包命名前缀规范是edu.nju.nlp.* 。对应的代码目录结构是edu->nju->nlp->*。
包名的后续部分根据不同机构各自内部的命名规范而不尽相同,NLP组的包后缀部分命名应同样遵守有意义、可理解的原则,具体要求可参照1.1节相关内容。

示例如下:
%\begin{enumerate}[(a)]
%\item  

(a) \textcolor{blue}{com.sun.eng }  
%\item  

(b) \textcolor{blue}{edu.nju.nlp.keyword}
%\end{enumerate}
\end{enumerate}

\subsection{C\#}

\begin{enumerate}
\item 名空间

名空间命名包括:CompanyName.ProjectName.Feature。其中,CompanyName:公司名称;Project Name:该项目缩写;Feature:实现的功能。每个单词的首字母需要大写。需要注意的是名空间和类不能使用同样的名字。NLP组的名空间中CompanyName为NJUNLP。
项目名和特征名的命名应同样遵守有意义、可理解的原则,具体要求可参照1.1节相关内容。

示例如下:
%\begin{enumerate}[(a)]
%\item  

(a) \textcolor{blue}{Microsoft.office.ui}
%\item  

(b) \textcolor{blue}{NJUNLP.Keyword.Extrator}
%\end{enumerate}
\end{enumerate}
