% !TEX TS-program = XeLaTeX
% !TEX encoding = UTF-8 Unicode

\chapter{项目规范}
\label{chap03}

项目规范包含项目管理、代码管理、文档管理等多个方面内容,我们使用redmine来管理自己的项目,使用git来管理自己的代码。

\section{项目管理}

\subsection{项目管理平台}

NLP组使用基于开源工具Redmine的项目管理平台,可以用于管理代码、文档并提供实用信息反馈、版本更新,并提供讨论区,代码评审等功能。组内同学定期必须往redmine上传代码,文档等。nlp关于redmine的使用请参考附录5.3。

\subsection{目录和文件}

代码和文档是项目的主要部分,因此代码和文档在项目中应该单独存放和管理。建议在每个项目中都包含src和doc两个目录,对应存放该项目的代码和文档文件。此外,部分项目还应该包含lib文件夹用于存放该项目所使用的工具包和程序库,data文件夹用以存放该项目相关的数据。需要特别注意的是,编译结果或可执行文件应单独存放,如Java中应选择用bin文件夹单独存放编译得到的二进制文件等

\subsection{编码}

需要注意的是,由于NLP组的工作涉及到大量的多种语言相互转换,文件编码往往成为许多bug的罪魁祸首。因此,我们要求在所有文件(包括源文件和数据文件)的管理中,默认使用utf-8编码。如有特殊情况需要使用其他编码的,应予以显式说明。

\subsection{数据}

项目的数据使用应该清晰明确,大规模的数据应按照版本单独存储,在项目中只需要注明使用数据的版本即可,尽量少使用重复的数据副本。在项目保存和提交过程中应随项目保留小规模的测试数据,用以验证程序的正确性。


\section{代码管理}

\subsection{版本控制}

NLP组统一使用开源的git工具进行代码的版本控制。关于git的使用请参考附录5.4

\subsection{代码测试}

任何一个项目都需要进行充分的测试,这包括最基本的项目能否正确的运行,到速度、效率上面的追求。

测试代码的最好方法是给代码写测试。写出来的测试可以反复的执行。当你修改了实现,可以通过再次运行测试来检查是否引入了任何bug。这种方法可以将你从调试中拯救出来,并且可以帮助你设计更好的代码。

NLP组的代码应该都经过较为完善的测试过程以保证正确性。

\subsection{代码审查}

代码审查即code review。代码审查(code review)可以检验代码设计的合理性(如实现方法,数据结构,设计模式,扩展性考虑等),是否存在大量重复代码等等。在代码审查中,参与审查的人员可以充分了解代码的设计和实现,方便后期的维护,也减少了项目风险。NLP组的重要项目代码应尽量保证进行过代码审查。

\section{文档管理}

软件开发过程中,文档有着非常重要的作用。通过文档代码阅读者能够迅速的理解程序设计思想,从而为程序改进或者二次开发帮助极大。这不仅节省时间同时也有利于该项目的可持续发展。完整的项目中必须包含设计文档,使用文档,更新记录、Readme以及版权信息。项目中的文档应包含下面几个方面的内容:

\subsection{设计文档}

\begin{enumerate}
\item 设计目的

描述项目实现的功能以及包含的主要代码文件、每个代码文件之间的关系;
\item 设计思路

描述项目中使用的主要算法;输入输出接口说明,以及在项目开发过程中使用到的参考文献;
\end{enumerate}

\subsection{使用文档}

使用文档记录介绍了项目的使用方法。
\subsection{更新记录}

里面记录了项目升级时的性能提升、新增的功能或者修正了上一个版本出现bug等等。
格式:[version][date]

示例如下:
\begin{lstlisting}
1.05     10-10-2010
* 修改了基本数据结构,优化了速度
* 增加了中文识别的功能
* 修改了版本1.0出现的bug
\end{lstlisting}


\subsection{Readme}

列出了项目开发过程中做出贡献的作者以及联系方式。

\subsection{版权信息}

版权信息的使用方法参见本文档的第4部分。

