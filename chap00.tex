% !TEX TS-program = XeLaTeX
% !TEX encoding = UTF-8 Unicode

\chapter*{\hfill 前  言 \hfill}
\addcontentsline{toc}{chapter}{前  言}
\label{chap00}

南京大学自然语言处理组(NJUNLP)在数十年的发展过程中积累了大量的与科研相关的代码、程序和工具,包含着每一届同学和每一位老师的辛勤工作的成果。但是,由于受到个人工作风格和人员流动等因素的影响,这些工作的传承和重用逐渐遇到了困难。为了更好的对这些工作进行管理和维护,我们制定了这一份编程规范。希望能够通过规范化的手段控制代码生成、文档管理、项目管理等过程,以及包括版权协议在内的诸多其他方面。

从开始编制到完成初稿历时超过一个月的时间,我们参考了大量的相关规范,针对组内的实际情况进行了组合、删减和优化,并给出了相应的示例;力求让这份规范能够符合NLP组的实际情况,并尽量减少实际执行过程中的负担。尽管删改多次,其中难免会有一些疏漏。如有任何不清楚、不恰当之处,请各位老师同学随时指出。我们一同管理和维护这份规范,为NLP组的发展尽一份力量。
请各位老师同学仔细阅读这份规范,并在工作中认真遵守规范中的相关约定。愿我们的工作不断进步。

本文档给出了南京大学自然语言处理研究组的开发过程中的需要遵守的命名规范、注释规范、项目管理规范以及组内项目所遵守的版权协议(包括但不限于开源协议)。本文的示例程序中默认使用的是GPL协议。附录中详细说明了Java和C\#中对文档注释的语言支持。

请严格执行本文档中的约定(少量可选的约定在文档中标记为optional)

~~~~~~~~~~~~~~~~~~~~~~~~~~~~~~~~~~~~~~~~~~~~~~~~~~~~~~~~~~~~~~~~~~~~~~~~~~~~~~~~~~~~~~~~~~~~~~~~~~~~~~~~~~黄书剑

~~~~~~~~~~~~~~~~~~~~~~~~~~~~~~~~~~~~~~~~~~~~~~~~~~~~~~~~~~~~~~~~~~~~~~~~~~~~~~~~~~~~~~~~~~~~~~~~~~~~~~~~2012~年~10~月~7~日