% !TEX TS-program = XeLaTeX
% !TEX encoding = UTF-8 Unicode

\chapter{注释规范}
\label{chap02}

类、方法、变量等命名意义明确的代码本身为理解代码打下了良好的基础,但实际工作中仍需要在代码中加入额外的解释和说明成分,这些成分应采用注释来完成。适当的注释对保证代码可读性、项目可持续发展至关重要。
代码中的注释分为实现注释、文档注释和文件注释三种类型。其中实现注释用于直接解释代码内容,文档注释用于具体说明代码的结构、文件注释用于说明版权,作者等信息。注释规范中约定了需要进行注释的内容和注释使用的格式。
注释中不应包括诸如制表符和回退符之类的特殊字符。

\section{实现注释规范}

\subsection{注释内容}

程序关键逻辑,特别对于实现代码中重要的地方,应采用实现注释进行说明

\subsection{注释格式}

实现注释的格式在不同语言中有所不同。例如:在perl、python等脚本语言中采用\#标记;在C++,Java和C\#等语言中可以使用//或/* */两种格式。下面详细说明C++、Java和C\#中的实现注释风格。脚本语言中的注释可参照短注释和行末注释。

C++、Java和C\#中的实现注释风格包括:块注释、短注释,行末注释,TODO注释。

\subsubsection{块注释}

块注释之首应该有一个空行,用于把块注释和代码分割开来.块注释通常用于提供对文件,方法,数据结构和算法的描述。块注释被置于每个文件的开始处以及每个方法之前。它们也可以被用于其他地方,比如方法内部。在功能和方法内部的块注释应该和它们所描述的代码具有一样的缩进格式。

示例如下:

\begin{lstlisting}[language=c]
/*
  Here is a block comment.
*/
\end{lstlisting}


\subsubsection{短注释}

短注释分为单行注释、尾端注释。它们可以在一行内对需要解释的代码给出注释。单行注释之前有一个空行并且需要与其后的代码具有一样的缩进层级。尾端注释需要与所要描述的代码位于同一行。

\begin{enumerate}
\item 单行注释,示例如下:

\begin{lstlisting}[language=c]
if (condition) {                      
    /* Handle the condition. */
 		...
}
\end{lstlisting}

\item 尾端注释,示例如下:

\begin{lstlisting}[language=c]
if (a == 2) {
    return TRUE;              /* special case */
} 
else {
    return isPrime(a);         /* works only for odd a */
}
\end{lstlisting}
\end{enumerate}

\subsubsection{行末注释}

注释界定符"//",可以注释掉整行或者一行中的一部分。它一般不用于连续多行的注释文本;然而,它可以用来注释掉连续多行的代码段。

示例如下:

\begin{lstlisting}[language=c]
if (foo > 1) {
    // Do a double-flip.
      	...
}
else {
    return false;          // Explain why here.
}
//if (bar > 1) {
  	//
  	//    // Do a triple-flip.
  	//    ...
  	//}
  	//else {
  	//    return false;
  	//}
\end{lstlisting}

\subsubsection{TODO注释}

对于那些临时的,短期的解决方案或者不够完美的代码使用TODO注释。TODO注释格式:[TODO :]

示例如下:

\begin{lstlisting}[language=java]
 Public SomeClass(){  
	 //TODO:Add Constructor Logic here
 }
\end{lstlisting}

\section{文件注释规范}

\subsection{注释内容}

在每个文件的开头应加入文件注释,用于说明版权、许可版本、作者、版本等信息。具体内容如下:

\textcolor{cyan}{版权:如Copyright © 2012 Nanjing university}

\textcolor{cyan}{许可版本:项目的使用许可信息,如GPL,LGPL,BSD,Apache2.0;NLP组的使用许可信息参见本文档第4部分}

\textcolor{cyan}{作者:文件的原始作者信息以及联系方式}

\textcolor{cyan}{版本:初始版本信息}

\textcolor{cyan}{文件描述:简要介绍文件的内容,文件中类的功能等,如果文件中包含多个类,还应描述这些类的相互关系}

\textcolor{cyan}{修改信息:记录文件修改的信息,包括修改日期,修改人,修改内容等}

\clearpage
\subsection{注释格式}

这里java,c,c++,c\#等语言以\textcolor{cyan}{/*}开头,以\textcolor{cyan}{*/}结尾。具体格式见下面。

至于python,perl,ruby等语言的文件注释只要包含上述文件注释内容,符合语言语法即可。
示例如下:
\begin{figure}[htpd]
 \centering
 \includegraphics[scale=0.8]{document-commentation.jpg}
 \caption{文件注释}
\end{figure}

\section{文档注释规范}

文档注释用于标明与文件的相关版权作者等信息和对代码的结构和内容进行系统性的解释和说明。
文档注释规范约定了NLP组的文档注释应该包含的内容,并对文档注释应该使用的格式进行了说明。

\subsection{注释内容}

下面从类、方法两个层次分别说明文档注释的内容。
\subsubsection{类注释}

每个类的定义之前应该加入类注释,用于对该类进行简单的介绍,并描述关于类的功能和用法。如果该类有与其他类进行交互的重要功能函数或接口函数,也应该在类注释中进行简要说明。

示例如下:

\begin{lstlisting}[language=java]
/** Description of MyClass
  * The function of MyClass is Blah Blah 	
  * The method of MyClass is Balh Blah
  */
public class Myclass {
}
\end{lstlisting}

\subsubsection{方法注释}

在关键方法(例如实现主要功能的方法、被多个类调用的方法、可能被重用的方法等)的定义之前应该加入方法注释.

用于描述以下内容:
\begin{enumerate}
\item 方法的功能和实现,包括该方法的主要功能,方法中使用的关键算法等信息。如有需要,还应该说明该方法可能产生的异常等;
\item 参数描述,包括需要传入的参数的意义、要求等
\item 返回值描述,返回值的意义、可能发生的异常情况等
\end{enumerate}
示例如下:

\begin{lstlisting}[language=java]
/** Description of myMethod(int a,double b)
  * The function of myMethod is Blah Blah
  * @param  a 	Description of a 
  * @param  b	Description of b
  * @return c	Description of c
  */
public object myMethod(int a ,double b){
    object c;
    // Blah Blah Blah...
    return c;
}
\end{lstlisting}

\subsection{注释格式}

文档注释应使用对应程序设计语言所提供的注释机制进行。例如,perl和python中使用\#符号,C++中使用//或/* */等。

特别的,Java和C\#中对文档注释进行了特殊的支持。在Java和C\#中,如果采用给定的格式进行文档注释,可以利用辅助工具将这些注释直接转化为html或者xml表示的文档,从而更加直观的得到对于代码信息的描述。因此,我们规定在Java和C\#的程序中,应该使用其内建的格式进行文档注释。下面简要介绍这两种语言中内建的文档注释的支持。


\subsubsection{Java文档注释}

Java中的文档注释内容由/**...*/界定。文档注释中可以通过插入标签来描述指定内容。标签使用如下形式描述:
\begin{lstlisting}
 @label description
\end{lstlisting}

其中label指的是具体的标签名,而description是对该标签的具体描述。Java中支持的标签包括version,author等(详细描述和示例见附录5.1)。文档注释可以通过Javadoc工具转换成HTML文件。

\subsubsection{C\#文档注释}

C\#中的文档注释内容由///界定。C\#提供的机制是可以使用含有XML文本的特殊注释语法为代码编写文档。XML标记可以使用如下形式描述:
\begin{lstlisting}
<label>description</label>
\end{lstlisting}

其中label指的是具体标记名称,description是对该标记的具体描述。C\#中支持的标签包括version,author等(详细描述和示例见附录5.2)。

C\#的文档注释可用Microsoft Visual Studio生成对应的XML文件。具体方法如下所示:1)在“解决方案资源管理器”中选定一个项目,然后在“项目”菜单中单击“属性”;2) 单击“生成”选项卡;3)在“生成”页上,选择“XML 文档文件”。默认情况下,在指定的输出路径(如“bin$\backslash$$\backslash$Debug$\backslash$$\backslash$Projectname.XML”)下创建该文件。












